\documentclass[12pt]{article}

\usepackage[utf8]{inputenc}
\usepackage[hidelinks]{hyperref}

\title{Week 1 Mini Paper}
\author{Dylan M Ang}
\date{August 4th, 2021}

% P1 is false, they do not need to be intrinsically reason giving.

\begin{document}

\maketitle

% Sentence 1: In your own words, concisely and accurately summarize a passage from the text.
In a bid to support the argument for moral skepticism, J.L. Mackie argues that objective moral facts would need to be incredibly weir and unlike any other type of rules, specifically that they would need to be intrinsically reason giving.
% Sentences 2-3: Tell me something about its role in the argument (Is it a premise? How does the author justify it? What conclusion does the author use it to argue for?). 
This is the first premise of Mackie's argument for queerness. He uses this to argue that since we have been unable to rigorously discern objective moral values, than it follows that there simply are no objective truths.
% Sentences 4-5: Raise an objection or question you would like the author to address. 
While I do subscribe to moral skepticism, I object to Mackie's first premise that objective moral facts would need to be intrinsically reason giving. I think it's totally plausible that moral facts could exist outside of our motivations to follow them, and that we will just never know what they are unless we create/discover a system that effectively tests the truth values of moral statements.

\end{document}
