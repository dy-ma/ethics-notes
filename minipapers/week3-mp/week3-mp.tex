\documentclass[12pt]{article}

\usepackage[utf8]{inputenc}
\usepackage[hidelinks]{hyperref}

\title{Week 3 Paper}
\author{Dylan M Ang}
\date{\today}

\begin{document}

\maketitle

% Sentence 1: In your own words, concisely and accurately summarize a passage from the text.
In \textit{Groundwork of the Metaphysics of Morals}, Immanuel Kant argues that suicide is immoral according to the second formulation of the categorical imperative because a suicidal person is simply using their corporeal person ``merely as a means to maintain a tolerable condition up to the end of life.'' 
% Sentences 2-3: Tell me something about its role in the argument (Is it a premise? How does the author justify it? What conclusion does the author use it to argue for?). 
This piece is an example Kant uses to illustrate how making judgements using the categorical imperative formulations. He likely picked it because he and his audience believed suicide to be an immoral action, so it is a familiar example and easy to judge.
% Sentences 4-5: Raise an objection or question you would like the author to address. 
However, I believe Kant is wrong about this, because suicidal people don't just live with no consideration for the betterment of their situation. Most people in general understand that they themselves are autonomous beings with their own aspirations and goals, but with suicidal people have great pain, or lack of will to live; that doesn't mean they don't see themselves as people. Using Kant's first formulation, suicide is moral because suicide becoming universal law would not hinder your ability to kill yourself. 

\end{document}