\documentclass[12pt]{article}

\usepackage[utf8]{inputenc}
\usepackage{xcolor}
\usepackage[hidelinks]{hyperref}

\newcommand{\ans}[1]{\textcolor{purple}{#1}}

\title{Week 3 Questions}
\author{Dylan Ang}
\date{August 24th, 2021}

\begin{document}

\maketitle

\begin{enumerate}
    \item Kant thinks that the good will is the only thing that is intrinsically valuable. Why does he think nothing else is intrinsically valuable?
          \ans{Kant considers two things as being intrinsically good--talents of mind and gifts of fortune. These both seem like good candidates at first, but Kant counters that these things are really only good if done by a good person, i.e., with good will. This makes sense, most people believe that good character traits and material goods are good, but would hardly want a bad person to come into posession of them. For this reason, Kant says that talents of mind and gifts of fortune are in fact not intrinsically valuables, but a good will is.}
    \item In one way, Kant's view looks a lot like rule consequentialism: according to both views, what is right and wrong depends on universal rules. Explain how the two views differ.
          \ans{In rule consequentialism, the best action is the one in accordance with the set of rules that maximizes welfare. However, Kant's universalizability test simply asks; if everyone did what you are about to for the same reason, could you still accomplish your goal? While both involve following rules, Kant doesn't say that your goal has to maximize welfare. It is much more personal--can you accomplish your goal? Your goal may not be what everyone elses goals are, but that's ok according to Kant. To a rule consequentalist, your actions aren't judged int he context of your own goals, they are analyzed in the frame of one universal goal, maximizing welfare.}
    \item Does Kant think it is morally better to help others out of a sense of duty, or because you are a naturally helpful person? Do you agree or disagree, and why?
          \ans{Kant thinks it is morally better to help others out of a sense of duty, rather than because you are a naturally helpful person. I understand where he's coming from, seems to me like if you are a naturally helpful person then helping others is, in a way, helping you. You're doing it because it makes you feel good about yourself. That being said, if I run this through the universalizability test, then I come to the conclusion that helping others has moral worth. In a world where everyone is naturally helpful and helps others, I could still be helpful and I would live comfortably. I don't necessarily think you have to sacrifice for it to be good. In addition, I don't think this violates the second formulation of the categorical imperative.}
    \item Pick an action you believe to be immoral and run it through the first formulation of the categorical imperative. Is the maxim universalizable? Explain your reasoning.
          \ans{I would consider stealing from a poor person to be immoral. So hypothetically I want to steal from a poor person. I consider, if everyone stole from poor people, there may not be enough poor people to steal from anymore, and I won't be able to steal anymore. In addition, I may be considered poor myself, and then people could steal from me. Therefore, stealing from a poor person is wrong.}
    \item Kant formulates the categorical imperative in three different ways. He famously thinks that they are three ways of saying basically the same thing. Identify the first two formulations, and tell me whether you think Kant is right that they're basically the same. Pick a specific action to illustrate your point (i.e., show that both formulations come to the same conclusion about it).
          \ans{I would agree that they might come across similar conclusions most of the time, but I don't think they are really the same. As an example (discussion of rape warning), I think rape is an immoral action. From Kant's second formulation, rape is clearly wrong if you consider human beings as rational and autonomous beings who deserve respect. Violating someones body without their consent is not respecting them, so it is wrong. From the perspective of the first formulation though, I'm not sure it's as concrete. If my goal is to rape someone for my own pleasure, and then that maxim becomes universalized, then everybody would rape for their pleasure. That doesn't necessarily mean that I will get raped myself, but even if I did, that doesn't stop me from accomplishing my goals. Maybe I am misunderstanding the criteria, but that's what I thought of.}
\end{enumerate}

\end{document}