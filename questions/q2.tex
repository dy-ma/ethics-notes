\documentclass[12pt]{article}

\usepackage[utf8]{inputenc}
\usepackage{xcolor}
\usepackage[hidelinks]{hyperref}

\newcommand{\ans}[1]{\textcolor{purple}{#1}}

\title{Week 2 Questions}
\author{Dylan M Ang}
\date{August 14, 2021}

\begin{document}

\maketitle

\begin{enumerate}
    \item Imagine you find out that your friend’s partner was cheating on them. The partner begs you not to tell your friend. If you tell your friend,they and their partner will break up and each of them will be heartbroken for months. If you do not tell them, they will go on to have a long, happy relationship. The partner will not cheat ever again. What would an act utilitarian advise you to do, and why?
    \ans{The act utilitarian would advise you to keep the secret for your friend's partner. They choose this path because act utilitarians only care about what choice will result in the best outcome, i.e. highest general welfare. Even if lying was considered morally wrong, it would be morally permissible because it resulted in the long and happy relationship, rather than the heartbreak.}

    \item Which objection that Mill considers do you think poses the most serious problem for act utilitarianism? Summarize it in your own words, and then, also in your own words,summarize Mill’s response. Is his response effective? Why or why not?
    \ans{I believe the best objection to act utilitarianism is that it's too demanding. The idea being that we would need to evaluate our choices as a disinterested spectator and determine what action would result in the highest well-being, which is too much effort and too difficult to do for most humans. Mill's defense is that most people should just follow the rules they already know, because our moral codes have usually been shaped by some form of utilitarianism. There are some problems with this defense, however. In the developed west, sure a lot of rules follow utilitarianism, but there are many people whose culture has nothing to do with utilitarianism. Those people would need to rethink a good portion of their moral codes. In addition, is Mill giving people a framework for evaluating moral actions and making moral decisions, and then proceeding to suggest they just ignore it when it becomes too difficult? Forgive me for thinking that that just doesn't make sense.}

    \item What is the difference between a criterion of rightness and a decision procedure?
    \ans{A criterion of rightness is a factor that we take into account when we want to determine the moral rightness of an action. Whereas a decision procedure is a method for making morally correct decisions. They are often similar and it can be difficult to discern which is which, but a criterion of rightness applies more to decisions that have already been made, and decision procedures apply more to decisions that you are about to make.}

    \item Briefly explain the difference between act consequentialism and rule consequentialism. Which do you find more attractive, and why?
    \ans{Act consequentialism says that the most morally correct action is the action that maximizes overall happiness. Rule consequentialism says that the most morally correct action is the one most in line with the set of rules that maximizes overall happiness. I think act consequentialism makes more sense, and is a better way to examine actions, but it's difficult to do. What I mean is, when making a decision, it's a lot easier to just follow the established rules, rather than try and figure out the best action. However, if we had the time, and the ability to see the impacts of our choices, then act consequentialism makes sense. Since we don't, I think rule consequentialism is overall more attractive.}

    \item What is one sense in which Hooker thinks act consequentialism is unreasonably demanding? Why does he think rule consequentialism fares better in this respect?
    \ans{Hooker argues that under act consequentialism, we would be required to donate the majority of our disposable income to those in need, take higher paying careers so we can donate more (even if we hate the job), and spend our free time volunteering, because these maximize overall well-being. To most people though, this just seems like too much. Rule consequentialism, he says, is more forgiving. We still donate, but the optimal rules might pick a much lower amount. One reason this could be true is that if people are forced to donate all of their money and work jobs they hate, they might just refuse to work for lack of incentives. So the rule consequentialists account for this and adjust accordingly.}
\end{enumerate}

\end{document}