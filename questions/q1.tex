\documentclass[12pt]{article}

\usepackage[utf8]{inputenc}
\usepackage{xcolor}
\usepackage[hidelinks]{hyperref}

\newcommand{\ans}[1]{\textcolor{purple}{#1}}

\title{Week 1 Questions}
\author{Dylan M Ang}
\date{August 11, 2021}

\begin{document}

\maketitle

Readings: Mackie, Prinz, Shafer-Landau

\begin{enumerate}
    \item What feature of moral facts and properties does Mackie think is so weird? Do you agree with Mackie that moral facts and properties would have to have this feature? If not, explain. If so, do you think this undermines the existence of moral facts and properties?
    \ans{Mackie beleives that objective moral facts would have to be weird in the sense that they would have to be self-motivating. I disagree with Mackie here in that I believe it seems possible for objective moral truths to exist independently of our motivations to follow them. Now, they would need to be self motivating in order for any human to actually be able to understand which moral judgements are objectively true and which are simply their own subjective ideals. Even widespread agreement from every culture does not necessarily imply a universal, objective rule. However, it isn't the case that objective moral facts must be self-motivating to exist, they would simply need to be self-motivating in order for humans to be able to understand and follow them.}

    \item What feature of moral discourse does Prinz think objectivists can’t adequately explain? What does he think is wrong with their explanations? Do you agree or disagree, and why? \ans{Prinz believes that moral objectivists can't adequately explain moral variation among different cultures. He categorizes popular responses as either denying variation or denying the importance of variation. Objectivists that deny variation typically claim that variation among cultures is mostly only variation in the appliance of the same basic values. However, Prinz argues that cultures that declare cannibalism and child marriage morally permissible are simply too different, and can not be seen as holding the same set of basic values as other cultures. The idea that variation doesn't actually matter is essentially that disagreements are natural and happen in science too, yet there is still an objective truth in science. So is the same in morality, some moral codes are objectively worse than others, and as we gain knowledge, we get closer to the objective truth. Prinz argues that morality doesn't improve like science. Whereas in science, we have methods to resolve disagreeements, there are none such in ethics. He points out that often when a moral ideal 'wins' out over another, it isn't because of logic or reason, it's because one society has dominated another. I have one objection to his first conclusion, which is that while cannibalism and child marriage may seem too different to be an interpretation of the same set of basic values, I don't think it's impossible. We really have no idea what, specifically, the moral truths are (if we are to presume they exist). They could be incredibly basic, in such a way that even cannibals share them, who are we to say? The other objection is to Prinz's second point, that objectivists deny that variation matters. Prinz's main counterargument is against the idea that the world's moral code has improved, that we haven't gotten any better. However, that isn't what was being argued. He's right that we haven't necessarily improved as a species, and that we don't have any way to resolve moral disputes, but he doesn't have a direct, sound argument against the idea that the different cultures with different values could simply be wrong. We may have no way to know exactly which, and we may have no way to scientifically resolve these issues, but that doesn't mean that some cultures could simply not hold the objective truths as values.
    }

    \item Most eighteenth-century Americans believed that it was permissible to enslave fellow human beings. The relativist cannot say that they were wrong. What, if anything, can the relativist say to make their position more palatable. Are you satisfied with this response? Why or why not? \ans{The relativist cannot claim that being anti-slavery is objectively more correct than being pro-slavery, but they can still argue for abolishment using other methods. One example could be finding common ground with supporters of slavery to argue that they have made a mistake in their moral judgements. In addition, one could argue that it is in their interests to be anti-slavery, that paying employees wages is cheaper than purchasing and housing slaves, or that the practice of slavery hinders scientific progress by holding back potentially great minds. Personally, I'm satisfied with this response. I've found that in my personal experience, trying to convince someone to change their mind of an issue they believe strongly in is very difficult. You need to use every appeal you can, even if you thought it was valid criticism, it won't work to simply claim that your ideals are morally superior and that everybody needs to conform to them unconditionally.}

    \item Explain the difference between objectivity, absolutism, universality as they apply to morality. \ans{Objectivity in relation to morality is the belief that there are objective moral facts, irregardless of people's feelings about them. However, moral objectivism still allows for taking circumstances into account. For example, even if murder is objectively wrong, it could also be objectively justifiable in the case of self defense. Moral absolutism is the idea that there are moral acts that are always wrong, no matter the context or circumstances surrounding them. For example, is murder is wrong absolutely, then murder is never justifiable, not even in self defense. Moral universality is the idea that some moral rules apply universally, to all people, no matter their culture. However, it is not necessarily absolutism, because it does not always imply that these universal rules are never justifiable, but only that they apply to everyone.}

    \item Could there be objective moral facts that were not absolute? Could there be objective moral facts that were not universal? Could there be moral facts that were absolute and universal, but not objective? Give an example of each (do not use examples from lecture). \ans{Philosopher Shafer-Landau believes that you can indeed have objective moral facts whicha re not absolute. For example, he mentions that lying is a sin, but there may be good reasons to lie that will be given an exception. I object that you can have objective moral facts which are not absolute. Any exception to the objective rules that would be indicative of non-absolutist rules would simply be a part of the objective rules. Just because the laws wouldn't be written into a stone tablet doesn't mean they aren't part of the rules. Suppose that murder is objectively wrong, but committing murder in self defense is considered morally permissible. Then, Shafer-Landau says, you have an objective moral rule that is not absolute. However, then it would simply be the case that the law is adjusted, murder is objectively wrong, except in the case of self defense. Now, the rule applies absolutely. Moving on, I do believe it is possible for objective laws that are not universal. An example of this could be: rich people should donate their wealth to better the world. This rule only applies to people with enough excess capital to do so, and it wouldn't be expected for people who are struggling to donate their next meal. Lastly, it is difficult to imagine a moral fact that is absolute and universal, but not objective. However, under the assumption that there are no objective moral facts, I believe it would be possible to find something that can apply to all people from any culture, but not because it is objectively correct, but simply because rational, well-intentioned people would all agree on it. One example of such a rule I can imagine is to seek not to cause unnecessary harm, because it makes our lives very uncomfortable.}

\end{enumerate}

\end{document}
