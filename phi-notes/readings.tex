\documentclass[12pt]{article}

\usepackage[utf8]{inputenc}
\usepackage[hidelinks]{hyperref}

\title{Readings Summary}
\author{Dylan M Ang}
\date{\today}

\begin{document}

\maketitle

\tableofcontents

\section{Mackie. Moral Skepticism}

\paragraph{The Argument from Relativity}
\begin{itemize}
    \item [P1] Different cultures often hold different normative ideas of what is considered moral.
    \item [P2] Considering that the differences are often vast, it is unlikely that they are simply a different interpretation of an objective truth.
    \item [C1] People draw their moral code around what they practice.
    \item [P3] Counter: Some people are reformers, and seek to change the moral standard. Reformers often draw the line in accordance to a previous belief.
    \item [C] There are no objective values.
\end{itemize}

\paragraph{The Argument from Queerness}
\begin{itemize}
    \item [P1] If there were moral facts that applied to us, they would have to be intrinsically reason giving.
    \item [P2] We are unable to rigorously discern any objective values.
    \item [C] There are no objective values.
\end{itemize}

\section{Prinz. Cultural Relativism}

\begin{itemize}
    \item [P1] Cultures around the world hold different ideas of what is considered moral, and engage with those behaviors differently.
    \item [P2] Concerning morality, there is no standard, no test, for truth.
    \item [P3] Parents use punishment (physical punishment, withdrawing love, ostracize, deprive, vicarious distress) to prevent certain behaviors in children.
    \item [P4] People often make moral judgements even with no good reason to.
    \item [C1] Moral judgments are often based in emotion.
    \item [C2] We cannot change basic values by reason alone. Reason can only convince someone that their basic values are in need of revision.
    \item [C] The source of our moral inclinations is merely cultural.
\end{itemize}

\section{Enoch. Moral Objectivism}

\paragraph{Objectivity Test}
\begin{itemize}
    \item Spinach Test
    \item Phenomenology of disagreement and deliberation test
    \item Counterfactual test
\end{itemize}

\section{Shafer-Landau. Moral Realism}

\paragraph{Objectivity Requires Absolutism}
\begin{itemize}
    \item [P1] If moral claims are objectively true, then moral rules are absolute.
    \item [P2] No moral rule is absolute.
    \item [C] Therefore, moral claims are not objectively true.
    \item [R1] It's not clear that are no absolute moral rules. Rape and deliberately killing innocents is pretty absolute.
    \item [R2] Even if there are no absolute moral rules, there can be objective moral facts which are not absolute.
\end{itemize}

\paragraph{All Truth is Subjective}
\begin{itemize}
    \item [P1] There are no objective truths.
    \item [C] Therefore, there are no objective moral truths.
    \item [R1] This argument is self-defeating. If P1 is true, then P1 is not objectively true.
\end{itemize}

\paragraph{Equal Rights Imply Equal Plausibility}
\begin{itemize}
    \item [P1] If everyone has an equal right to an opinion, then all opinions are equally plausible.
    \item [P2] Everyone has an equal right to their moral opinions.
    \item [P3] Therefore, all moral opinions are equally plausible.
    \item [P4] If all moral opinions are equally plausible, then ethical objectivism is false.
    \item [C] Therefore, ethical objectivism is false.
    \item [R1] The fact that we have an equal right to our opinions has nothing to do with the plausibility of our opinions.
\end{itemize}

\paragraph{Moral Objectivity Supports Dogmatism}
\begin{itemize}
    \item [P1] If there are objective moral standards, this makes dogmatism acceptable.
    \item [P2] Dogmatism is unacceptable.
    \item [C] Therefore, there are no objective moral facts.
    \item [R1] Objectivity supports humility. If there are objective moral facts, then we can get them wrong, and the right attitude to take when we are fallible is humility.
\end{itemize}

\paragraph{Moral Objectivity Supports Intolerance}
\begin{itemize}
    \item [P1] Tolerance is valuable only if the moral views of different people are equally plausible.
    \item [P2] If ethical objectivism is true, then the moral views of different people are not equally plausible.
    \item [C] Therefore, if ethical objectivism is true, then tolerance is not valuable.
    \item [R1] P2 is true, but P1 is false. Objectivism is actually a lot better than non-objectivism at supporting tolerance.
    \item [P1] If all moral views are equally plausible, then moral views that support tolerance and those that support repression and intolerance are equally plausible.
    \item [P2] Moral views that support repression and intolerance are not equally as plausible as those that support tolerance.
    \item [C] Therefore, not all moral views are equally plausible.
\end{itemize}

\paragraph{Moral Objectivity Cannot Allow for Legitimate Cultural Variation}
\begin{itemize}
    \item [P1] If moral objectivism is true, then all moral standards apply universally.
    \item [P2] Some moral standards do not apply universally, but only in certain cultural contexts and not others.
    \item [C] Therefore, ethical objectivism is false.
    \item [R1] P2 is true, but P1 is false. Objectivists don't need to think that all moral standards are universal.
    \item [R2] Objectivists tend to think basic moral principles are universal, but different societies might interpret that differently.
\end{itemize}

\paragraph{Moral Disagreement Undermines Moral Objectivity}
\begin{itemize}
    \item [P1] If rational people disagree about some claim, then that claim is not objectively true.
    \item [P2] Rational people persistently disagree about all moral claims.
    \item [C] Therefore, no moral claim is objectively true.
    \item [R1] It's not obvious that P2 is true. Disagreement about moral claims could be the result of ignorance, sloppy reasoning.
    \item [R2] P1 is false. Scientists disagree about objective facts all the time.
\end{itemize}

\paragraph{Atheism Undermines Moral Objectivity}
\begin{itemize}
    \item [P1] Morality can be objective only if God exists.
    \item [P2] God doesn't exist.
    \item [C] Therefore, morality cannot be objective.
    \item [R1] P1 is false, since it rests on the assumption that laws require lawmakers.
    \item [R2] If you are an atheist, then you presumably believe in laws of logic, physics, chemistry, statistics, that you believe in but were not created by a lawmaker.
\end{itemize}

\paragraph{The Absence of Categorical Reasons Undermines Objectivity}
\begin{itemize}
    \item [P1] If there are objective moral rules, then there are categorical reasons to obey them.
    \item [P2] There are no categorical reasons.
    \item [C] Therefore, there are no objective moral duties.
    \item [R1] It's not clear that either premise is true.
    \item [R2] There could be objective moral standards even if whether we have reasons to obey them depends on what we care about.
    \item [R3] It could be that we sometimes have reasons to do things even when there's nothing in it for us.
\end{itemize}

\paragraph{Moral Motivation Undermines Moral Objectivity}
\begin{itemize}
    \item [P1] Moral judgements are able, all by themselves, to motivate those who make them.
    \item [P2] Beliefs are never able, all by themselves, to motivate those who make them.
    \item [P3] Therefore, moral judgements are not beliefs.
    \item [P4] If moral judgements are not beliefs, then they can't be true.
    \item [C] Therefore, moral judgements can't be true.
    \item [R1] We can reject P1 or P2. Some objectivists deny that moral judgements motivate us all by themselves. Sure, our moral judgements usually motivate us, but that's because we care about morality. If we didn't want to be moral, then maybe our moral judgements wouldn't motivate us. Moral motivation = moral judgment + desire to be moral.
    \item [R2] Reject P2. Most beliefs can't motivate us on their own, but evaluative beliefs, i.e., those about what is good, bad, right, wrong, rational, etc. can.
\end{itemize}

\paragraph{Values Have No Place in a Scientific World}
\begin{itemize}
    \item [P1] If science cannot verify the existence of X, then the best evidence tells us that X does no exist.
    \item [P2] Science cannot verify the existence of objective moral values.
    \item [C] Therefore, the best evidence tells us that objective moral values do not exist.
    \item [R1] Naturalistic moral realists reject P2. Naturalists hold that moral facts and properties are ordinary facts and properties that we can find out about empirically.
    \item [R2] Non-naturalistic moral realists reject P1. non-naturalists hold that while science can tell us about a lot, it can't tell us about everything. Moral facts and properties are among those science can't tell us about.
\end{itemize}

\section{Mill. Act Utilitarianism}

The right action is always the one that maximizes overall well-being.

\paragraph{Objections and Responses}
\begin{itemize}
    \item Utilitarians are only concerned about their own happiness.
          \begin{itemize}
              \item Utilitarianism requires us to act like ``disinterested spectators'' - our own interests count exactly as much as every other person's.
          \end{itemize}
    \item Utilitarianism is too demanding. It's not possible to care more about the general happiness than our own interests.
          \begin{itemize}
              \item Utilitarianism is a theory about how you should act, not what should motivate you to act that way. Utilitarians only care about the consequences of the act, not the motivation.
          \end{itemize}
    \item Character matters. Saving a drowning stranger out of duty is better than for a reward.
          \begin{itemize}
              \item Sure, that person is a better person. But Utilitarianism is supposed to give us a standard for evaluating actions, not people.
          \end{itemize}
    \item Utilitarianism seems to be in tension with religion - it makes morality about our happiness instead of God's commands
          \begin{itemize}
              \item If God is omni-benevolent (as Abrahamic religions tend to assume) then he would want to promote the general happiness. Also gives us a reason for why God set these rules.
          \end{itemize}
    \item Utilitarians only are about what is useful.
          \begin{itemize}
              \item It's concerned with what promotes the general happiness, not just us or our group.
          \end{itemize}
    \item Utilitarianism is too demanding. Every time we do anything, it requires us to sit down and evaluate each action.
          \begin{itemize}
              \item People think of morality all the time, follow the rules you already know. Utilitarianism is an explanation of why these principles are true.
          \end{itemize}
\end{itemize}

\section{Hooker. Rule Consequentialist}
Rule Consequentialists believe that you should follow the rules, if the rule is generally good.

\section{Terms}

\noindent

\textbf{Moral Skepticism:} The position that knowledge of any objective morality is impossible.

\textbf{Moral Subjectivism:} The position that ethical propositions are true or false based on the attitudes of people.

\textbf{Hypothetical Imperative:} If you want X, do Y.

\textbf{Categorical Imperative:} You should do Y.

\textbf{Moral Relativism:} The position that multiple moral beliefs can be simultaneously true.

\textbf{Emotional Osmosis:} Child see, child do.

\textbf{Intrinsic Value:} Value for its own sake, rather than as a means to an end.

\textbf{Absolutism:} An absolute moral rule is one that is always wrong to break.

\textbf{Consequentialism:} The rightness (and wrongness) of an action depends only on the value of its consequences.

\textbf{Maximization:} You should always take the action that produces maximally valuable consequences.

\textbf{Welfarism:} Welfare/well-begin is the only intrinsically valuable thing.

\textbf{Aggregationsim:} The value of an outcome is determined by adding up the total (everyone's) welfare in that outcome.

\textbf{Act Utilitarianism:} In any scenario, the right action is the one that maximizes overall well-being.

\textbf{Hedonism:} Well-begin consists in happiness/pleasure.

\textbf{Hedonistic Act Utilitarianism:} The right action is always the one that maximizes overall happiness/pleasure.

\textbf{Rule Consequentialism:} The right action is the one that is done in accordance with the system of rules which, if generally followed, would produce the best consequences.

\end{document}