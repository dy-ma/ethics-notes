\documentclass[12pt]{article}

\usepackage[utf8]{inputenc}
\usepackage[hidelinks]{hyperref}

\title{Lecture Notes}
\author{Dylan M Ang}
\date{August 2nd, 2021}

\begin{document}

\maketitle

\tableofcontents

\section{Events}
\begin{itemize}
    \item Review Syllabus
    \item Discuss Philosophy Powerpoint
\end{itemize}

\section{Notes}

An argument has two parts:
\begin{itemize}
    \item Conclusion: What is being argued.
    \item Premises: The defense of the conclusion.
    \item Watch out for background info. Support for the premises, but not necessary.
\end{itemize}

\paragraph{Standard Form:}
\begin{itemize}
    \item Condense items into a list featuring the premises and the conclusion.
\end{itemize}

\paragraph{Types of Ethics}
\begin{itemize}
    \item Normative Ethics: Questions how we should live our lives.
    \item Applied Ethics: Concerned with the application of normative ethical principles to moral problems.
    \item Metaethics: Concerned with the nature of moral thought and talk. Metaethicists ask what morality is.
    \begin{itemize}
        \item Metaphysics: What exists and what are those things like? 
        \item Moral Metaphysics: Are there moral facts? If so, what makes them true? If not, why do people think there are?
        \item Epistemology: What is knowledge? When is a belief justified?
        \item Moral Epistemology? How do we figure out what actions are right?
        \item Philosophy of Language: In virtue of what do words mean things?
    \end{itemize}
\end{itemize}

\paragraph{Realism vs. Anti-Realism}

Moral realism is a view about moral metaphysics. It is usually said to consist of three claims. Moral realism is the same as moral objectivism.
\begin{enumerate}
    \item Moral judgements are truth-apt. (Cognitivism)
    \item Some moral judgements are true. (Success Theory)
    \item Some moral judgements are objectively true. (Objectivism)
\end{enumerate}

Moral Anti-realism is the view that there are no objective moral facts.
\begin{itemize}
    \item Deny objectivism. Moral facts can be true or false, but they are not objectively so.
\end{itemize}

Anti-realists can disagree with one or all of the three claims made from moral realists.

\subsection{Error Theory}

\begin{itemize}
    \item [V] Cognitivism
    \item [V] Success Theory
    \item [X] Objectivism
\end{itemize}

There is no inherent good or bad. Moral judgements can not have a true/false value.

\subsection{Categorical vs Hypothetical Imperatives}

\noindent

\textbf{Hypothetical Imperatives:} If you want X, you should do Y.

\textbf{Categorical Imperatives:} You should do Y.

\end{document}
